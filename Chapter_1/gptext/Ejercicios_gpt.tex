\documentclass{article}
\usepackage{exam}

\begin{document}

\section{Introducción}

Este es un ejemplo de documento TeX.

\subsection{Matemáticas}

La ecuación de la recta es $y = mx + b$.

\[\sum_{i=1}^{n} i^2 = \frac{n(n+1)(2n+1)}{6}\]

\section{Ejercicios de C, A partir del ejercicio de Fahrenheit}

Aquí tienes cuatro ejercicios más desafiantes basados en el contenido:

\subsection{Ejercicio 1}
Escribe un programa en C que calcule la suma de los números enteros de un arreglo y luego imprima el resultado. Sin embargo, esta vez, el arreglo contendrá números enteros negativos y el programa debe ignorarlos al calcular la suma.

\subsection{Ejercicio 2}
Crea un programa en C que pida al usuario que ingrese una temperatura en grados Fahrenheit y luego la convierta a grados Celsius. El programa debe utilizar la fórmula correcta para la conversión y redondear el resultado a dos decimales.

\subsection{Ejercicio 3}
Escribe un programa en C que imprima la tabla de multiplicar de un número entero ingresado por el usuario. La tabla debe mostrar los productos del número entero multiplicado por los números enteros del 1 al 10.

\subsection{Ejercicio 4}
Crea un programa en C que calcule el promedio de una serie de números enteros ingresados por el usuario. El programa debe pedir al usuario que ingrese la cantidad de números que desea ingresar y luego leer cada número. Finalmente, debe calcular y imprimir el promedio de los números ingresados.

\end{document}